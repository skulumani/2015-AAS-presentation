% slides discussing variational integrator
\section*{}
\subsection*{Variational Integrator}
\begin{frame} %------------------------------------%
\frametitle{Variational Principle}
	\begin{columns}[c]
		\begin{column}{0.5\textwidth}
			\centering
			\begin{beamercolorbox}[wd=0.8\columnwidth,sep=0.05cm,center]{numerical} Continuous Time \end{beamercolorbox}
			\begin{beamercolorbox}[wd=0.8\columnwidth,sep=0.05cm,center]{numerical} 
				Configuration Space \\
				\( \parenth{q, \dot{q} } \in TQ \)
			\end{beamercolorbox}
			\begin{beamercolorbox}[wd=0.8\columnwidth,sep=0.05cm,center]{numerical} 
				Lagrangian \\
				\( L\parenth{q, \dot{q} } \)
			\end{beamercolorbox}
			\begin{beamercolorbox}[wd=0.8\columnwidth,sep=0.05cm,center]{numerical} 
				Action Integral \\
				\( S = \int_{0}^T L\left( q, \dot{q}\right) \, dt \)
			\end{beamercolorbox}
			\begin{beamercolorbox}[wd=0.8\columnwidth,sep=0.05cm,center]{numerical} 
				Stationary Action \\
				\( \delta S = 0 \)
			\end{beamercolorbox}
%			\begin{beamercolorbox}[wd=0.8\columnwidth,sep=0.05cm,center]{numerical} 
%				Legendre Transform \\
%				\( p_i = \deriv{L}{\dot{q}} \)
%			\end{beamercolorbox}
			\begin{beamercolorbox}[wd=0.8\columnwidth,sep=0.05cm,center]{numerical} 
				Equation of Motion \\
				\( \dot{q} = f \parenth{q, \dot{q} } \)
			\end{beamercolorbox}
		\end{column}
		\begin{column}{0.5\textwidth}
			\centering
			\begin{beamercolorbox}[wd=0.8\columnwidth,sep=0.05cm,center]{numerical} Discrete Time \end{beamercolorbox}
			\begin{beamercolorbox}[wd=0.8\columnwidth,sep=0.05cm,center]{numerical} 
				Configuration Space \\
				\( \parenth{q_k, q_{k+1} } \in Q \times Q \)
			\end{beamercolorbox}
			\begin{beamercolorbox}[wd=0.8\columnwidth,sep=0.05cm,center]{numerical} 
				Lagrangian \\
				\( L_d\parenth{q_k, q_{k+1}} \)
			\end{beamercolorbox}
			\begin{beamercolorbox}[wd=0.8\columnwidth,sep=0.05cm,center]{numerical} 
				Action Sum \\
				\( S_d = \sum_{k=0}^{N-1} L_d(q_k, q_{k+1}) \)
			\end{beamercolorbox}
			\begin{beamercolorbox}[wd=0.8\columnwidth,sep=0.05cm,center]{numerical} 
				Stationary Action \\
				\( \delta S_d = 0 \)
			\end{beamercolorbox}
%			\begin{beamercolorbox}[wd=0.8\columnwidth,sep=0.05cm,center]{numerical} 
%				Fiber Derivative \\
%				\( p_k = - \deriv{L_d(q_k, q_{k+1})}{q_k} \) \\
%				\( p_{k+1} = \deriv{L_d(q_k, q_{k+1})}{q_{k+1}} \)
%			\end{beamercolorbox}
			\begin{beamercolorbox}[wd=0.8\columnwidth,sep=0.05cm,center]{numerical} 
				Equation of Motion \\
				\( q_{k+2} = f_d \parenth{q_k, q_{k+1} } \)
			\end{beamercolorbox}
		\end{column}
	\end{columns}

\end{frame}%-----------------------------------------%

\begin{frame}[shrink=20] %------------------------------------------%
\frametitle{Discrete Lagrangian}
	Continuous time Lagrangian
	\begin{align*}
		L = \frac{1}{2} \left( \left( \dot{x} -y \right)^2 + \left( \dot{y} + x \right)^2 \right) + \frac{1-\mu}{r_1} + \frac{\mu}{r_2}
	\end{align*}
	Choice of Quadrature rule affects accuracy
	\begin{table}[htbp]
		\centering
		\begin{tabular}{l|l}Rectangle & \( L_d(q_0,q_1) =L(q_0,\frac{q_1-q_0}{h}) h \)  \\ \hline
		Midpoint & \( L_d(q_0,q_1) = L(\frac{q_0 + q_1}{2},\frac{q_1 - q_0}{h}) h \) \\ \hline
		Trapezoidal & \( L_d(q_0, q_1) = \frac{1}{2} \left[ L(q_0, \frac{q_1 - q_0}{h} ) + L(q_1, \frac{q_1 - q_0 }{h} )\right] h \)
		\end{tabular} 
	\end{table}
	Discrete time Lagrangian using Trapezoidal Rule
	\begin{align*}
		L_d &= \frac{h}{2} \left( \frac{1}{2} \bracket{\left(  \frac{\xkp - \xk}{h} -\yk \right)^2 + \left( \frac{\ykp - \yk}{h} + \xk \right)^2} + \frac{1 - \mu}{r_{1_k}} + \frac{\mu}{r_{2_k}} \right. \nonumber \\ 
	&+ \left. \frac{1}{2} \bracket{\left(  \frac{\xkp - \xk}{h} -\ykp \right)^2 + \left( \frac{\ykp - \yk}{h} + \xkp \right)^2} + \frac{1-\mu}{r_{1_{k+1}}} + \frac{\mu}{r_{2_{k+1}}}  \right)
	\end{align*}
	
\end{frame} %--------------------------------------------%

\begin{frame} %---------------------------------------------------%
\frametitle{Integrator Comparison}
	\begin{itemize}
		\item Numerical Simulation of PCRTBP (\( t_f = 200 \approx 15 \) years)
			\begin{itemize}
				\item Variable step Runge-Kutta method (\texttt{ode45.m})
				\item Variational integrator \( \bar{x}_k \to \bar{x}_{k+1} \)
			\end{itemize}
		\item Energy drift behavior 
	\end{itemize}
\begin{figure} 
	\centering 
	\begin{subfigure}[h]{0.5\textwidth} 
		\includegraphics[width=\textwidth]{./integrator_compare/trajectory} 
	\end{subfigure}~ %add desired spacing between images, e. g. ~, \quad, \qquad, \hfill etc. %(or a blank line to force the subfigure onto a new line) 
	\begin{subfigure}[htbp]{0.5\textwidth} 
		\includegraphics[width=\textwidth]{./integrator_compare/energy} 
	\end{subfigure} 
	\end{figure}

\end{frame} %------------------------------------------------------%