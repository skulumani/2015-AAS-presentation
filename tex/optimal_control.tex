\section*{}
\subsection*{Optimal Control}
\begin{frame} %-----------------------------%
\frametitle{Reachability Set}
  \begin{itemize}
  \item Reachable set on Poincar\'e section: the set of states that can be attained from a given initial state via admissable control input
  		\begin{itemize}
  			\item Enlarge the intersection region on the Poincar\'e section
  		\end{itemize}
  \item \emph{Computational Geometric Optimal Control}
	\begin{itemize}
  		\item Poincar\'e section defined by \( \alpha_d \) in \( m_1\)
		\item Direction on Poincar\'e section defined by \( \theta_d \) in \( m_2 \)
	\end{itemize}
 \end{itemize}
  \begin{align*}
	J &= -\frac{1}{2} \left( \bar{x}(N) - \bar{x}_{n}(N)\right)^T Q_f\left( \bar{x}(N) - \bar{x}_{n}(N)\right)\\
	m_1 &= 0 = \frac{y(N) - L_{1y}}{x(N) - L_{1x}} - \tan{\alpha_d} \\ 
    m_2&= 0 = \frac{\dot{x}(N) - \dot{x_n}(N) }{x(N) -x_n(N) } - \tan{\theta_d} \\
	 0 &\geq\bar{u}^T \bar{u} - u_{max}^2 
	\end{align*}

\end{frame}   %-----------------------------%

%\begin{frame}%--------------------------------------------%
%\frametitle{Euler-Lagrange Equations}
%\begin{itemize}
%	\item Necessary conditions for optimality are used to develop a two-point boundary value problem
%	\item Reachable set on Poincar\'e section is approximated by varying \(\theta_d\)
%\end{itemize}
%\begin{subequations}\label{eq:necc_cond}
%\begin{align*}
%	\lambda_{k+1}^T &= \lambda_k^T  \deriv{f_k}{\bar{x}_k}^{-1} \\
%	0 &=  \deriv{H_k}{\bar{u}_k} \\
%	0 &= \deriv{\phi}{\bar{x}_k}^T + \deriv{m}{\bar{x}_k}^T\beta  - \lambda^T(N) 
%\end{align*}
%\end{subequations}
%\end{frame} %-------------------------------------------------%