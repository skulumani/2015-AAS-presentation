% slides discussing the invariant manifolds and introducing poincare sections
\section*{}
\subsection*{Invariant Manifolds}

\begin{frame}%--------------------------------------------%
\frametitle{Invariant Manifolds}
\begin{itemize}
	\item Tube structure governs orbital motion
	\item \textcolor{green}{Stable}/\textcolor{red}{Unstable} `tubes' divide orbits
	\item Poincar\`e section visualizes intersection
\end{itemize}
 \begin{figure}
     \centering
        \begin{subfigure}[b]{0.5\textwidth}
                \includegraphics[width=\columnwidth]{U2_Manifolds}
        \end{subfigure}%
        ~%add desired spacing between images, e. g. ~, \quad, \qquad, \hfill etc.
          %(or a blank line to force the subfigure onto a new line)
        \begin{subfigure}[b]{0.5\textwidth}
                \includegraphics[width=\columnwidth]{U3_Manifolds}
        \end{subfigure}
\end{figure}
\end{frame}%--------------------------------------------%

\begin{frame}[t]%---------------------------------------%
\frametitle{Poincar\'e Section}
	\begin{itemize}
		\item Define hyperplane transverse to dynamic flow
		\item Analysis on a lower dimensional space
	\end{itemize}

    \includegraphics[width=0.5\columnwidth]{U2_Manifolds}~
    \includegraphics[width=0.5\columnwidth]{U2_poincare}

	\pause
	\begin{beamercolorbox}[sep=0.2cm,center]{numerical}
		Transfer limited to intersecting regions \\
		Control is required 
	\end{beamercolorbox}
\end{frame}%--------------------------------%

